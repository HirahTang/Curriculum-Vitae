%%%%%%%%%%%%%%%%%%%%%%%%%%%%%%%%%%%%%%%%%
% Medium Length Professional CV
% LaTeX Template
% Version 2.0 (8/5/13)
%
% This template has been downloaded from:
% http://www.LaTeXTemplates.com
%
% Original author:
% Rishi Shah 
%
% Important note:
% This template requires the resume.cls file to be in the same directory as the
% .tex file. The resume.cls file provides the resume style used for structuring the
% document.
%
%%%%%%%%%%%%%%%%%%%%%%%%%%%%%%%%%%%%%%%%%

%----------------------------------------------------------------------------------------
%    PACKAGES AND OTHER DOCUMENT CONFIGURATIONS
%----------------------------------------------------------------------------------------

\documentclass{resume} % Use the custom resume.cls style

\usepackage[left=0.75in,top=0.6in,right=0.75in,bottom=0.6in]{geometry} % Document margins
\usepackage{hyperref}
\newcommand{\tab}[1]{\hspace{.2667\textwidth}\rlap{#1}}
\newcommand{\itab}[1]{\hspace{0em}\rlap{#1}}
\name{Han Tang} % Your name


%\address{123 Pleasant Lane \\ City, State 12345} % Your secondary addess (optional)

\address{16 Victoria Ave, Donnybrook, Dublin 4 D04 T2P1} % Your address
\address{(+353)0830477054 \\ D16129273@mytudublin.ie} % Your phone number and email

\begin{document}

\centerline{\href{http://www.linkedin.com/in/han-tang-308a4117a}{LinkedIn Profile: Han Tang}}


%----------------------------------------------------------------------------------------
%    EDUCATION SECTION
%----------------------------------------------------------------------------------------

\begin{rSection}{Education & Training}

{\bf Technological University of Dublin, Kevin St., Dublin, Ireland} \hfill {\em September 2018 - Present} 
\\ MSc in Computing (Data Analytics)
\\ School of Computing\\
\\Award expected: 2.1

Relevant Subjects: 
\\Working With Data (A): Programming in SQL and R.
\\Machine Learning (A\textsuperscript{-}),  Deep Learning (A): Machine Learning in Python.
\\Data Mining (B\textsuperscript{+}): The full life cycle of Data Mining project deployment.
\\ {\bf Dissertation:}
\\A Comparison Study on State-of-the-art Minority Class Data Oversampling Techniques for Imbalanced Learning
\\

\\{\bf Dublin Institute of Technology, Kevin St., Dublin, Ireland} \hfill {\em September 2017 - June 2018} 
\\ Pre-master for MSc in computing. \hfill {\em Average score: 78}

%Minor in Linguistics \smallskip \\
%Member of Eta Kappa Nu \\
%Member of Upsilon Pi Epsilon \\

\\{\bf Beijing University of Chemical Technology, Beijing, China} \hfill {\em September 2013 - June 2017} 
\\ BSc in Applied Chemistry
\\Dissertation:
\\Study the factors of the layer heights of Layered Double Hydroxides.
\end{rSection}


%--------------------------------------------------------------------------------
%    Projects And Seminars
%-----------------------------------------------------------------------------------------------
\begin{rSection}{Projects}

\\{\bf Dissertation: A Comparison Study on State-of-the-art Minority Class Data Oversampling Techniques for Imbalanced Learning }

\href{https://github.com/HirahTang/MScDissertation_Han_Tang}{\bf Github Link: Msc Dissertation Han Tang}

Study the state-of-the-art approaches to imbalanced learning. The focus of study is on Synthetic Minority Oversampling Technique (SMOTE) and its extensions.

Compare more than ten data resampling methods' performances on 35 datasets, and check whether their differences are in a statistic level.



\\
{\bf Propose a data security system for the German Government}

Group work: Analyse the security risks in the German government Data Management System and propose measures to minimize them.

As an individual: Establish relevant policies and implementation plans to enforce the security measures.

\\{\bf Multi-label Bird Species Classification - NIPS 2013 Kaggle Competition}

\href{https://github.com/HirahTang/NIPS-2013-Bird-song-classification}{\bf Github Link: NIPS 2013 bird song classification}

Produce Deep Learning Neural Networks to identify the bird species from their records, trained from a dataset contains 1000 instances and 87 classes.

Plot the Mel-frequency spectrograms of the soundtracks, then to train convolutional neural networks from the spectrogram graphs.

The architecture is from the ConvNet in the package Fastai. The result of AUC is 0.83.




%\\{\bf Develop machine learning and object recognition models from the Street View House Number (SVHN) Dataset}

%\\Predictive analyse the Street View House Number (SVHN) image dataset, produce Convolutional Neural Networks and reaches an average accuracy of 95\%.


% Add the SQL model on Portuguese bank dataset and that of Data visualisation set (Tableau and Python)


\end{rSection}

\begin{rSection}{Writings}

\\{\bf Effective feature engineering techniques on data set contains sequential feature}

\href{https://github.com/HirahTang/Ford-Challenge}{\bf Github Link: Ford Challenge}

The data set retrieved from Kaggle competition - ’Stay Alert! The Ford Challenge’ records the change of drivers' behaviours over time.

Rolling means and standard deviations of each feature are introduced as new features to record the sequential change.

This feature engineering technique improves the performances of models of several machine learning algorithms, proved to be useful.


\end{rSection}
%----------------------------------------------------------------------------------------
%    TECHNICAL STRENGTHS SECTION
%----------------------------------------------------------------------------------------

%----------------------------------------------------------------------------------------
%    WORK EXPERIENCE SECTION
%----------------------------------------------------------------------------------------

\begin{rSection}{Career Experience}

\begin{rSubsection}{Postal Savings Bank of China}{July 2019 - September 2019}{Data Mining Analyst Intern.}

\item Conduct data mining, data modelling, statistical analysis. Data analytics support for customer credit risk assessment.

\item Arrange, edit and archive user guidance for frequently used machine learning packages such as Scikit-learn, Numpy, Pandas, TensorFlow, and Keras.

\end{rSubsection}

\begin{rSubsection}{State Key Laboratory of Chemical Resource Engineering}{November 2015 - May 2017}{Researcher}{}
\item Research on methods of calculating/estimating chemical parameters of complex compounds.
\item Calculate chemical parameters of compounds automatically in Python.
\end{rSubsection}



\end{rSection}

\begin{rSection}{Languages \& Skills}

\begin{tabular}{ @{} >{\bfseries}l @{\hspace{6ex}} l }
Technical Skills & SQL, MS Excel, R\\ & Python(4 years), Machine Learning\\ & Deep Learning, Data Mining\\ & LaTex, Tableau, SAS \\ & Keras, Pandas, Numpy \\ & Scikit-learn, TensorFlow\\ & Web Scraping \\

Soft Skills & Analytical Skills, Productivity\\ & Problem Solving, Teamwork\\ & Presentation Skills, Integrity\\ & Critical Thinking,  Creativity \\ 

Languages &  English (Full professional proficiency), Mandarin (Native proficiency)\\
\end{tabular}
\end{rSection}
%----------------------------------------------------------------------------------------
% Extra Curricular
%----------------------------------------------------------------------------------------

%\item Trained and disciplined in National Cadet Corps (NCC), IIT Kanpur for a year.
 %\item  Participated in Vijyoshi Camp 2012 organized at Indian Institute of Science, Bangalore.
 %\item Won 2nd position in Kho-Kho in Intramurals conducted by Physical Education Section, IIT Kanpur.
 %\item Pursued French as a second language during secondary school from Grade 6 to Grade 10. Also participated in French Song Competition and French G.K. Quiz in Class 10th. %



\begin{rSection}{Interests}
\begin{tabular}{ @{} >{\bfseries}l @{\hspace{6ex}} l }
Associations: &  Member of UK Oracle User Group\\
& \href{http://github.com/HirahTang}{Github Page: http://github.com/HirahTang}.\\

Other activities: & Play in a weekend football amateur league. \\ & Enthusiastic in mathematics and general science.
\end{tabular}
\end{rSection}
\end{document}
